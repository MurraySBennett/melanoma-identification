\documentclass[a4paper, natbib, doc, 12pt]{apa7}

% Packages
    \usepackage[T1]{fontenc}
    \usepackage[utf8]{inputenc}
    \usepackage{geometry}
    \geometry{
    left=1in,
    right=1in,
    bottom=1.5in,}

    \usepackage[english]{babel}
    \usepackage{amsmath}
    \usepackage{graphicx}
    \usepackage{url}
    \usepackage{natbib}
    \usepackage{authblk}
    % \usepackage[usenames,dvipsnames]{xcolor}
    % \usepackage{setspace}
    % \usepackage{hyperref}

    % \usepackage{fancyhdr}
    % \pagestyle{fancy}
    % \fancyhead[L]{MELANOMA PERCEPTUAL FEATURES}
    % \fancyhead[R]{}
    % \fancyfoot[C]{\thepage}

    % \usepackage{lmodern}
    % \usepackage[draft, obeyFinal, bordercolor=gray, backgroundcolor=white, linecolor=red]{todonotes}

    % Replace 'draft' with 'final' to remove all comments.
    \newcommand{\Note}[1]{\todo[inline]{#1}}
    % \usepackage{xargs}

    \interfootnotelinepenalty=10000
    \raggedbottom

    \def\linkosf{\rm{\url{https://appropriate-link.com}}}

\title{Human Perceptual Judgements and Computer Vision Assessments of Melanoma Features}
\shorttitle{}
\author{Murray S. Bennett}
\authorsnames{Murray~S.~Bennett, Joseph~W.~Houpt}
\authorsaffiliations{Department of Psychology, The University of Texas at San Antonio}

% \authornote{Correspondence should be sent to Murray S. Bennett, Department of Psychology, The University of Texas at San Antonio, Texas, United States. Email may be sent to murray.bennett@utsa.edu. Code for reproducing the analysis can be found at: \linkosf.}

\abstract{Melanoma is the deadliest form of skin cancer. Early detection is critical for improving survival rates, but accurate identification depends on subjective perceptual judgements made by human observers. Rules-based approaches to melanoma identification help reduce perceptual subjectivity by providing observers with specific perceptual features to evaluate skin lesions, thereby increasing diagnostic accuracy. The ABCD method, for example, directs observers to evaluate a skin lesion's shape asymmetry (A), border irregularity (B), colour variation (C), and diameter (D) when determining a diagnosis. However, these perceptual features differ greatly between lesions and are not necessarily present in all melanoma cases.
}


% \abstract{Melanoma is a deadly skin cancer, and early detection is critical for improving survival rates. Dermatologists typically rely on a visual scan to diagnose melanoma by assessing the primary perceptual characteristics of a skin lesion. The common ABCDE heuristic, for example, suggests observers check a lesion for shape (A)symmetry, (B)order irregularity, number of unique (C)olours, and (E)volution over time. Whilst this heuristic provides a practical guide, it is a limited approach. Firstly, all lesions vary and often contain only a subset of these features. Secondly, a combination of abnormal features can lead to a diagnosis, making the diagnostic process complicated and error-prone. Advanced computer vision algorithms (CVA) have emerged as a powerful approach to melanoma identification. CVAs can evaluate lesion features to generate highly accurate and objective assessments. However, despite CVA advancements, they can only be used in conjunction with an expert assessment. Thus, the perceptual expertise of dermatologists remains a critical component in the accurate and timely detection of melanoma. Our project aims to improve the early detection of melanoma by investigating the perceptual judgements of skin lesion colour and shape made by humans and comparing them with the feature representations generated by computer vision algorithms. We recruited non-expert participants online to complete a two-alternative forced-choice task using skin lesion images from the ISIC archive. Participants were instructed to choose the image that exhibited a greater frequency of unique colours in one condition and greater border regularity in another among the two images presented in a trial. We analysed the data using the Bradley-Terry-Luce (BTL) model to estimate each lesion image's relative "strengths" along these perceptual dimensions. We then compared these estimates to computer vision assessments of the same perceptual features. We discuss the methodological approach, preliminary results, and future directions.}

\keywords{computer vision, visual perception, melanoma identification, two-alternative forced choice, Bradley Terry Luce model}

\begin{document}
\maketitle

\setlength{\parskip}{0pt}


\section{Significance statement}

% \newpage
% \section{Research Question / Aim}
% Do human perceptual judgements of skin lesion features differ from computer vision algorithms?
% How are the perceptual features of skin lesions combined to generate diagnostic decisions of melanoma?
% How are the perceptual features of skin lesions integrated when diagnosing melanoma?
% How does expertise affect the integration of perceptual information in melanoma diagnosis?

% Do human perceptual judgements of skin lesion asymmetry, border regularity, and colour variability differ to computer vision algorithm assessments?

% \section{What we did}
% We conducted a series of 2AFC tasks where pairs of skin lesions were presented side-by-side for participants to select the image they perceived to be greater along a prompted perceptual dimension. Specifically, we asked participants to evaluate images according to lesion symmetry, border regularity, and colour variance. We applied the Bradley-Terry-Luce model to these perceptual judgements to produce a scaled ranking of the relative perceptual strengths of each image along each of the assessed perceptual dimensions. We then applied computer vision algorithms to produce objective assessments of these features, and compared these assessments to human judgements.

% \section{What we found}


% \section{What it means}


% \section{Take-Home message}


% \section{Future direction}
% Complete data collection on the current dimension, and expand to other valued dimensions: colour, size

% More advanced computer vision assessment for the evaluation of features and for the segmentation of lesions.

% This currently represents a descriptive examination of a single perceptual dimension. Our ambition is to begin an evaluation of features when combined. For example, we could readily expect the perception of border regularity to be affected by lesion size. Whilst we don't have the ground truth for lesion size (e.g., two masks may be the same size because the photographer zooms in, when in reality the two lesions are quite different), we can still provide a proof of concept here by relating the mask size to the BTL scaling.

% Actual relationship to melanoma identification -- currently only looking at perception of real-world stimuli.

% How does a GRT representation actually map to human perception?


\newpage


\section{Introduction}
Skin cancer is the most common cancer in the United States \citep{guy2015vital}, Australia, and New Zealand \citep{perera2015incidence}, and melanoma is the most fatal form of skin cancer with an estimated $97, 000$ new cases of melanoma and $8, 000$ deaths in 2023 in the United States alone \citep{siegel2023cancer}. Melanoma is readily treatable by surgery and survival rates are high if detected in the early stages of growth \citep{davis2019current, rigel2010evolution}, but accurate and early identification in the early stages of melanoma growth is difficult as diagnosis depends upon subjective observations of visual cues \citep{dreiseitl2012differences}. Rules-based approaches to melanoma diagnosis are used to aid naive observers by providing guides to evaluate key perceptual features of skin lesions. The ABCDE approach, for example, directs observers to evaluate a lesion for  asymmetry, border irregularity, colour variation, diameter, and evolution over time. Such approaches raise awareness of key perceptual indicators of melanoma and improve diagnostic accuracy for naive observers \citep{rourke2015learning,tsao2015early}. However, such approaches have notable limitations. First, skin lesions, whether malignant or benign, are natural stimuli that vary substantially along these perceptual dimensions. Malignant melanoma may present with a subset of features (only a combination of ABCDE features might be necessary to alert the observer) and benign lesions may present with multiple features \citep{rigel2010evolution}. Additionally, whilst the ABCDE approach is rules-based, there is no clear distinction or benchmark to distinguish when a lesion's border has become notably asymmetrical or irregular and whilst there are more concrete guides for colour variation (more than 2 distinct colours within the lesion), the perceptual problem persists as, particularly when lesions are small, the distinction between unique colours can be difficult.

Finally, distinct features such as colour perception and size are not perceptually independent. That is, features overlap and are integrated.

A rules-based representation of perceptual features is unlike the perceptual processes used by experts \citep{norman1989development, gachon2005first}.


Early detection is imperative: the lives of melanoma patients depend on accurate and early diagnosis. Physicians often rely on personal experience and evaluate each patients lesions on a case-by-case basis by taking into account the patient's local lesion patterns in comparison to that of the entire body \citep{gachon2005first}. 

Clinical diagnosis accuracy for melanoma detection is between $x$ and $y\%$. Diagnostic accuracy varies with expertise -- expert observers, unaided by computer-based technologies, can attain as high as $x\%$ accuracy. Whereas inexperienced professionals, such as early career dermatologists or medical students, achieve $y\%$, and naive observers approximately $z\%$. 


Awareness of the ABCD criteria can improve the detection and classification accuracy of laypersons \citep{robinson2006skills}.

Many studies have examined changes in idnetification accuracy following ABCD criteria education, but the evaluation of these perceptual features is a highly dubjective endeavour. 

Both \cite{zanotto2011visual} and \cite{laskaris2010fuzzy} had participants rate specific perceptual features of skin lesions. Namely, each lesion was rated on a 0-10 scale for colour darkness, colour uniformity, asymmetry, border regularity, and roughness of the lesion's texture. \cite{gunasti2008interrater} and \cite{} had groups of participants of varying expertise rate lesion images for border irregularity on a 1-7 ordinal scale. 

\cite{robinson2006skills} had groups of naive participants undergo dermatologist guided ABCD training then rate lesion images along a 1-5 scale for asymmetry, border regularity, and colour evenness separately. Ratings were scored as correct if they were in agreement with the dermatologists' ratings.


These all represent subjective rating scales.

\section{Methods}
\subsection{Participants}
A total of 310 naive observers were recruited via the University of Texas at San Antonio (UTSA) undergraduate participant pool ($N= $) and the online recruitment platform Prolific ($N= $). Participants were rewarded with course credit or at a reward rate of $\$6$ per hour depending on participation source. Participants were allocated randomly to one of the six conditions upon entry to the experiment ($n$ per condition). Demographic information was not collected. The study was approved by the IRB at UTSA (FY22-23-82). 

% With $10, 000$ images, power analysis indicated that we would need $40, 500$ trials for a $98\%$ probability of returning a fully connected graph. We use 400 trials per participant, which translated to $101.25 = 102$ participants per condition. If we reverse-score and collapse matching conditions (e.g., border regularity and border irregularity), we need $51$ participants per condition. At roughly $3\$$ per participant, you're looking at $306 * \$3 = \$918$ for Prolific, plus whatever Pavlovia wants for their server. However, you can also offset some of this with UTSA SONA participants.

\subsection{Design}
Data were collected across six between-subject conditions. Each condition was represented by the stimulus feature prompted to the participant. Participants were prompted to judge image-pairs according to one of three stimulus features (shape asymmetry, border irregularity, and colour variation), where each feature was also prompted under a reverse-wording condition (i.e., shape symmetry, border regularity, and colour uniformity). Unique image identities, the selected image, and the decision response times were recorded for each trial. 

% The final data analysis was conducted on $x_{1}$, $x_{2}$, $x_{3}$, $x_{4}$, $x_{5}$, and $x_{6}$ participants in the symmetrical, asymmetrical,  consistent border, irregular border, colourful, and uniform colour conditions. 

\subsection{Materials}
Images were selected from the International Skin Imaging Collaboration archive (ISIC). The ISIC archive contained approximately $71, 671$ images at time of retrieval (February, 2023). Metadata that included melanoma status, unique lesion identification numbers, and various patient demographic information accompanied all images. Duplicate images were removed ($n = 860$; see \cite{cassidy2022analysis}). We subjected the remaining $70, 611$ images to our computer-vision pipeline (see Analysis section), thus producing computer-vision measures of asymmetry, border regularity, and colour variance for all images. We then ranked the images and sorted according to the computer-vision assessments of border regularity. Images with border regularity scores greater than 0.9 or less than 0.1 were excluded. All $n$ remaining skin lesions with a melanoma diagnosis were included for the experiment sample, and the remaining $10,000 - n$ images were selected from the ranked image set via random-normal sampling to produce a final stimulus set size of $10, 000$ images. Images presented during the experiment were rotated as required to present the longest image border along the horizontal axis then resized to $512\times384$px. 

\subsection{Procedure}

\subsection{Analysis}


% I would suggest googling some of these functions to help clarify the process. You should also generate a separate document that specifies this procedure, and ensure that it's recreatable. Further, upload this to OSF.

% 1. hair removal
%     a. grayscale
%     b. blackhat morphological filtering -- (morphologyEx(cv.MORPH_BLACKHAT,...)
%     c. closing -- morphologyEx(cv.MORPH_CLOSE,..)
%     d. gaussian blurring
%     e. intensify the idnetified hair/artefact contours for inpainting
%     f. inpaint -- cv.inpaint(..,..,.., cv.INPAINT_TELEA)
% 2. colour clustering
%     a. cv.medianBlur()
%     b. kmeans cluster the image to 7 clusters.
% 3. clahe colour transformation -- contrast limited adaptive histogram equalisation
%     a.  cv.createCLAHE
%     b. convert to HSV
%     c. apply clahe to each colour layer (H, S, and V)
%     d. recombine layers and convert back to BGR colour space.
% 4. otsu segment
%     a. otsu_thresh
%         i. convert BGR to HSV (3d. not required)
%         ii. use only the saturation and value layers.
%         iii. apply gaussian blur
%         iv. threshold the image using cv.threshold(img, 0, 255, cv.THRESH_BINARY + cv.THRESH_OTSU)
%         v. canny edge detection -- cv.Canny(threshold, 100, 200)
%         vi. dilate the contours to 'smooth' -- cv.dilate(canny_edge, kernel, iters=1)
%     b. extract largest contour
%     c. retrieve contour mask

\subsubsection{Bradley-Terry-Luce Model}%
\label{ssub:Bradley-Terry-Luce Model}


\section{Results}


\section{Discussion}


















\subsection*{Acknowledgements}
\subsection*{Availability of Materials}
\newpage
\bibliography{refs.bib}

\end{document}

